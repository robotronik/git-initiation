\documentclass[10pt,a9paper,handout]{beamer}
\usepackage[utf8]{inputenc}
\usepackage[francais]{babel}
\usepackage[T1]{fontenc}
\usepackage{amsmath,amsfonts,amssymb,tikz,colortbl,lmodern,xspace,subfigure}
\usepackage{tikz}

\definecolor{c979797}{RGB}{151,151,151}
\definecolor{cff0000}{RGB}{255,0,0}
\definecolor{c0000ff}{RGB}{0,0,255}

%\usefonttheme{serif}

\usetheme{Berlin}
\usecolortheme{beaver}
\setbeamertemplate{caption}{\raggedright\insertcaption\par}
\setbeamercolor{item projected}{bg=darkred,fg=white}
\beamertemplateballitem



\title{\textsc{Introduction à Git}}

\author{Club Robotronik Phelma}
\date{10 Décembre 2015}


\defbeamertemplate*{footline}{myfootline}
{
  \leavevmode%
  \hbox{%
  \begin{beamercolorbox}[wd=.32\paperwidth,ht=2.25ex,dp=1ex,left]{title in head/foot}%
    \hspace{4px}Félix Piédallu
  \end{beamercolorbox}%
  \hspace*{-5px}
  \begin{beamercolorbox}[wd=.34\paperwidth,ht=2.25ex,dp=1ex,center]{title in head/foot}%
    \insertshortauthor
  \end{beamercolorbox}%
  \hspace*{-5px}
  \begin{beamercolorbox}[wd=.36\paperwidth,ht=2.25ex,dp=1ex,right]{title in head/foot}%
    \insertshortdate{}\hspace*{2em}
    \insertframenumber{} / \inserttotalframenumber\hspace*{2ex} 
  \end{beamercolorbox}}%
  \vskip0pt%
}

\begin{document}
\setcounter{tocdepth}{1}
\begin{frame} \titlepage       \end{frame}
\begin{frame} \tableofcontents \end{frame}

\section{Ceci est une section en français}
\subsection{Et une sous-section traduite elle-aussi.}
\begin{frame}
Pour la première fois sur master, il y a du texte en français ! (Eh oui, la branche felix/localisation n'a pas modifié master !)
\end{frame}

\section{}
\subsection{}
\begin{frame}
\end{frame}

\section{}
\subsection{}
\begin{frame}

\end{frame}

\section{}
\subsection{}
\begin{frame}
\end{frame}


\begin{frame}
    \frametitle{\textsc{À vous de jouer  !}}
    \begin{center}
    \Large \textbf{C'est tout pour aujourd'hui !\\ N'hésitez pas à poser vos questions aux anciens ;) }
    \end{center}
\end{frame}
\end{document}
